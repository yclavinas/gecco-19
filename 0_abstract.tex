Multi-objective Evolutionary Algorithm based on Decomposition, MOEA/D, decomposes multi-objective problems into single-objective subproblems. All subproblems are treated uniformly and there is no priority among them. Each subproblem is related to an area of the theoretical Pareto Front. It is expected that some areas would be more difficult to approximate than others, leading to an unbalanced exploration of the search space. To balance exploration "Resource Allocation" techniques that prioritize certain subproblems have been proposed. In this paper we investigate the priority functions used to decide  which subproblems a should receive more resources. We propose that using diversity as the priority criteria has a positive effect on resource allocation. We consider four different priority functions: Diversity on objective space, diversity on decision space, relative improvement (Used on MOEA/D-DRA), and a random priority function for control. WE conducted an experimental analysis on the DTLZ and UF benchmark functions and on the lunar lading real-problem and compared the famous MOEA/D-DE variant with each four priority functions and without it any. The results indicate XXX.