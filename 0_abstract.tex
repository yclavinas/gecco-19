Multi-objective Evolutionary Algorithm based on Decomposition, MOEA/D, decomposes multi-objective problems into single-objective subproblems. All subproblems are uniformly treated and there is no priority among them. Each subproblem is related to an area of the theoretical Pareto Front. It is expected that different areas would be more difficult to approximate than others, leading to an unbalanced exploration of the search space. To balance exploration, "Resource Allocation" techniques that prioritize certain subproblems were proposed. Here we investigate how priority functions relate to MOEA/D in terms of performance. We consider four different methods as priority functions: diversity on the objective space, diversity in the decision space, a priority function with random values and the relative improvement, from MOEA/D-DRA. We conducted an experimental analysis on the DTLZ and UF benchmark problems and on the lunar landing real-world problem and compared the famous MOEA/D-DE variant with each four priority functions and without any. The results indicate XXX.