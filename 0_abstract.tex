The key characteristic of the Multi-Objective Evolutionary Algorithm Based on
Decomposition (MOEA/D) is that the multi-objective problem is decomposed into
multiple single-objective subproblems.
%
In standard MOEA/D, all subproblems receive the same computational effort.
However, as each subproblem relates to different areas of the objective space,
it is expected that some subproblems are more difficult than others.
%
Resource Allocation techniques allocates computational effort proportional to
each subproblem's difficulty. This difficulty is estimated by a priority
function. Using Resource Allocation, MOEA/D could spend less effort on easier
subproblems and more on harders ones, improving efficiency.
%
In this paper, we investigate different priority functions. We propose that
using diversity as the priority criteria results in better allocation of
computational effort.
%
We propose two new priority functions: objective space diversity and decision
space diversity.
%
We compare the proposed diversity based priority with previous approaches on the
DTLZ and UF benchmarks, as well as on a real world problem about selecting a
landing site for lunar exploration.
%
The proposed decision space priority achieved high HV and IGD values,
excellent rate of non-dominated solutions on the benchmark problems,
and highest rate of feasible solutions among all priority functions in
the severely constrained lunar exploration problem.
