\section{Conclusion}

Using priority functions do help

Using norm helps more

We expected that MRDL would help a lot, but it barely helped. Maybe more iteractions are needed.

%In this paper, we proposed a new utility function based on a geometrical perspective for MOEA/D, and the resultant algorithm is called MOEA/D-RAD. This algorithm determines the computational resource assigned to each subproblem, based on its contribution to the overall diversity of the population.
%
%We have compared the new approach with the MOEA/D-DE and a variant with generalized resource allocation strategy, MOEA/D-DRA, and the experimental results suggested our method performs well on many test problems.
%
%As mentioned before in this study, there are two ways of dealing with an unbalanced exploration of the search space: one is to allocate resources based on an utility functions while the second is to modify the behavior of the algorithm adaptively. %We understand that these two forms of addressing the same problem are not conflicting, therefore, in future works, we may explore using utility functions as a feedback information for adaptation strategies.
%
%A further, more extensive study could assess the effects of utility functions on MOEA/D-RAD by better understanding its relationship with the others components of the MOEA/D framework.