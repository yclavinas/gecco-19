\section{Discussion}
%Restating the aims of the study
The aim of the present research was to investigate how priority functions relate to MOEA/D. We proposed two new priority functions (related to diversity) for estimating difficulty and for calculating priorities among subproblems for better Resource Allocation. We \emph{isolated} the priority functions in MOEA/D as the only variant. This allowed us to examine their effect on the performance of MOEA/D.

These two new priority functions focus on different aspects of diversity. The first, MRDL, addresses diversity on the objective space while the second, the Norm, addresses diversity on the decision space. We then compared these new priority functions with the most popular approach, the Relative Improvement, and the standard MOEA/D.

This study has shown that using Norm as priority function effectively improves the performance of MOEA/D, since it achieved high HV and IGD values and excellent rates of non-dominated solutions on the benchmark problems. It also lead to the highest rate of feasible solutions among all priority functions in the severely constrained lunar exploration. Some of these results were superior than the results of the R.I. (specially the rate of non-dominated solutions). These results indicate that Norm indeed leads to more diversity of the final solution set, demonstrating the effectiveness of it as a priority function and as a direct way to increase diversity in MOEA/D. This suggests that it really there is a role for diversity in promoting better performance in HV and IGD metrics as well as higher rates of non-dominated solutions. In contrast, it seems MRDL performed just slightly better than MOEA/D and it did not generate enough difference to serve as an effective priority function. However, given the surprising results of Random, we infer that there is still space for finding more appropriate priority functions.

Overall, the findings of this work strengthens the idea that exploring priority function focusing on critical issues (such as diversity and proportion of non-dominated solutions) is worth of attention. This suggests the choice of priority functions is a critical component of a Resource Allocation system. Our results recommend R.I. or Norm as reasonable choices for Resource Allocation depending on the MOP being addressed.


In this work, we do not yet consider archive based Resource Allocation and archive based priority functions, such as MOEA/D-CRA~\cite{kang2018collaborative}. We will address this issue in a continuation to this study. There are many components and variants of MOEA/D and is interesting to combine the Norm priority function with the them. Then, we can further explore the relationship of priority functions based on diversity with others components and variants of the MOEA/D framework. How to define more efficient and effective utility functions for different problems is also worth further investigation (such as priority function that also consider constraints) as well as to verify the results of using priority function in other real-world problems.
%
