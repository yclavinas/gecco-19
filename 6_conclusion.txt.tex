\section{Discussion}
%Restating the aims of the study
The aim of the present research was to investigate how priority functions relate to MOEA/D. We proposed two new priority functions (related to diversity) for estimating difficulty and for calculating priorities among subproblems for better Resource Allocation. We \emph{isolated} the priority functions in MOEA/D as the only variant. This allowed us to effectively examine their effect on the performance of MOEA/D.

These two new priority functions focus on different aspects of diversity. The first, the MRDL, addresses diversity on the objective space while the second, the Norm, addresses diversity on the decision space. We then compared these new priority functions with the most popular approach, the Relative Improvement, and the standard MOEA/D.

This study has shown that using Norm as priority function effectively improves the performance of MOEA/D, since it achieved high HV and IGD values, excellent rates of non-dominated solutions on the benchmark problems. It also lead to the highest rate of feasible solutions among all priority functions in the severely constrained lunar exploration. Some of these results were superior than the results of the R.I. (specially the rate of non-dominated solutions). These results indicate that Norm indeed leads to more diversity of the final solution set, demonstrating the effectiveness of it as a priority function and as a direct way to increase diversity in MOEA/D. This suggests that it really there is a role for diversity in promoting better performance in HV and IGD metrics as well as higher rates of non-dominated solutions.  

In contrast, MRDL performed just slightly better than MOEA/D. We hypothesize that the reason for these results is that MRDL measures the diversity of a solution against all the population. It is in our understanding that other priority functions that consider diversity in the decision space should be studied with the goal of answering this hypothesis. 

%The scope of this study was limited to MOEA/Ds without archive populations. Therefore, priority functions that concentrate resources to subproblems that generates more non-dominated solutions to the archive population were not considered in this study~\cite{cai2015external},~\cite{kang2018collaborative}. 

Overall, the findings of this work strengthens the idea that exploring priority function focusing on critical issues (such as diversity and rate of non-dominated solutions) is worth (worthy?) of attention. This suggests that using only priority functions can be very effective for better Resource Allocation. We also confirmed that R.I., a common priority function from the literature, can be a good choice depending on the MOP being addressed. However, given the surprising results of Random, we infer that there is still space for finding more appropriate priority functions.


Our findings complement recent studies that addressed Resource Allocation with priority functions. For example, we extend the study of Zhou and Zhang in MOEA/D-GRA\cite{zhou2016all}, by exploring diversity in priority functions, and we carried out investigations on priority functions using only one population, in the contrary of the work of Kang et al. in MOEA/D-CRA~\cite{kang2018collaborative}. Our study certainly contributes to the literature, since we contribute to the understanding that diversity as priority functions are a simple yet efficient mechanism for improving the performance of MOEA/D as well as the rate of feasible and non-dominated solutions. 
 
%not important, may be left out
There are many components and variants of MOEA/D and is interesting to combine the Norm priority function with the them. Then, we can further explore the relationship of priority functions based on diversity with the others components and variants of the MOEA/D framework. How to define more efficient and effective utility functions for different problems is also worth further investigation (such as priority function that also consider constraints) as well as verify the results of using priority function in other real-world problems.
%
