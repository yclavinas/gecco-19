\section{Experimental Design}

%\begin{table}[!t]
%	\centering
%\begin{tabular}{@{}|l|l|@{}}
%		\toprule
%		\textbf{Parameters}   & \textbf{Values}          \\ \midrule
%		Initial value $u$     & 0.5, for every subproblem \\ 
%		Population size       & 150                      \\
%		Neighborhood size T & 20 \\ 
%		$\delta_p$ & 0.9 \\ 
%		$\phi$ & 0.5 \\ 
%		$\eta_m$ & 20 \\
%		$p_m$ & 0.03333333 \\
%		$n_r$ & 2 \\
%		\midrule
%		Number of evaluations & 60000 		\\		
%		Number of repetitions & 21                  \\ \bottomrule
%
%	\end{tabular}
%\vspace{1em}
%\caption{Parameter settings.}
%\label{table1}
%\end{table}


The question that we want to answer is how MOEA/D-DE performs when combined with: no priority function (none) and the priority functions 2-norm, MRDL, relative improvement and random. To answer that question we apply MOEA/-DE with the 4 priority functions to two artificial benchmark problems and the Lunar Landing real-world problem and we compared the performance against the results of the MOEA/D-DE. We compared the variants of priority functions using the DTLZ benchmark functions from~\cite{deb2005scalable} with 100 dimensions and $k =$ dimensions - number of objectives $+ 1$, where the number of objectives is $2$. We also consider the UF benchmark functions from~\cite{zhang2008multiobjective}, also with 100 dimensions. 


%\subsubsection{Real World Benchmark Function}


The Lunar Landing problem is a real-world problem that simulates the selection of landing sites for lunar landers~\citep{MoonOrbitingSatellite2015}. In lunar exploration plan, finding suitable landing site of the rovers has a very important function. Good landing sites ensure enough sunshine to provide energy for the rovers power supply while being in a region with scientifically interesting materials with little difficulties to the exploration. This is a minimization problem in which the two decision variables are the longitude and latitude with the objectives being: the number of continuous shade days, the total communication time (in reality, this is a maximization problem that was inverted with the goal of consistency), and the inclination angles. Although the number of design variables is small as two (latitude and longitude), it is considered to be a severe constrained problem due to the presence of two craters. In values, the two constraints are defined as continuous shade days being $ < 0.05$ while inclination angles being $<0.3$

%\subsubsection{Parameter Settings}
For every combination, the parameters are as follows. The population size $N = 350$, the update size $nr = 2$, the neighborhood size $T = 20$, and the neighborhood search probability $\delta_p = 0.9$. The DE mutation operator value is $phi=0.5$. The Polynomial mutation operator values are $\eta_m 20$ and $p_m = 0.03333333$~\cite{campelo2018moeadr}. The number of executions is 21. At each execution, the number of functions evaluations is 70000. For the Lunar Landing, we highlight that differ from the settings for the artificial benchmarks. Since this problems is a severe constrained one, we chose the population size $N = 5050$ and the number of functions evaluations is 60000. 

We perform statistical tests on the hypervolume (HV) metric values and Inverted Generational Distance (IGD) for measuring the quality of a set of obtained non-dominated solutions found by the algorithms on the, DTLZ and UF benchmark problem. Before calculating the HV value, the objective function was scaled between $0$ and $1$. The reference point for the HV calculation was set to $(1, 1)$. For the real-world Lunar Landing problem, we only perform statistical tests on the hypervolume (HV) metric values and the reference point for the HV calculation was set to $(1, 0, 1)$. Higher values of the HV indicate better approximations while lower values of the IGD indicate better approximations. In order to verify any statistical difference in the average performance given the different algorithms, the Pairwise Wilcoxon Rank Sum Tests was used, with confidence interval $\alpha = 0.05$ and with the Hommel adjustment method. For reproducibility the code is made available at XXX.

